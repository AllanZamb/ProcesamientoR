% Options for packages loaded elsewhere
\PassOptionsToPackage{unicode}{hyperref}
\PassOptionsToPackage{hyphens}{url}
%
\documentclass[
]{article}
\usepackage{lmodern}
\usepackage{amssymb,amsmath}
\usepackage{ifxetex,ifluatex}
\ifnum 0\ifxetex 1\fi\ifluatex 1\fi=0 % if pdftex
  \usepackage[T1]{fontenc}
  \usepackage[utf8]{inputenc}
  \usepackage{textcomp} % provide euro and other symbols
\else % if luatex or xetex
  \usepackage{unicode-math}
  \defaultfontfeatures{Scale=MatchLowercase}
  \defaultfontfeatures[\rmfamily]{Ligatures=TeX,Scale=1}
\fi
% Use upquote if available, for straight quotes in verbatim environments
\IfFileExists{upquote.sty}{\usepackage{upquote}}{}
\IfFileExists{microtype.sty}{% use microtype if available
  \usepackage[]{microtype}
  \UseMicrotypeSet[protrusion]{basicmath} % disable protrusion for tt fonts
}{}
\makeatletter
\@ifundefined{KOMAClassName}{% if non-KOMA class
  \IfFileExists{parskip.sty}{%
    \usepackage{parskip}
  }{% else
    \setlength{\parindent}{0pt}
    \setlength{\parskip}{6pt plus 2pt minus 1pt}}
}{% if KOMA class
  \KOMAoptions{parskip=half}}
\makeatother
\usepackage{xcolor}
\IfFileExists{xurl.sty}{\usepackage{xurl}}{} % add URL line breaks if available
\IfFileExists{bookmark.sty}{\usepackage{bookmark}}{\usepackage{hyperref}}
\hypersetup{
  pdftitle={Practica\_8.R},
  pdfauthor={LENGUAJE-R},
  hidelinks,
  pdfcreator={LaTeX via pandoc}}
\urlstyle{same} % disable monospaced font for URLs
\usepackage[margin=1in]{geometry}
\usepackage{color}
\usepackage{fancyvrb}
\newcommand{\VerbBar}{|}
\newcommand{\VERB}{\Verb[commandchars=\\\{\}]}
\DefineVerbatimEnvironment{Highlighting}{Verbatim}{commandchars=\\\{\}}
% Add ',fontsize=\small' for more characters per line
\usepackage{framed}
\definecolor{shadecolor}{RGB}{248,248,248}
\newenvironment{Shaded}{\begin{snugshade}}{\end{snugshade}}
\newcommand{\AlertTok}[1]{\textcolor[rgb]{0.94,0.16,0.16}{#1}}
\newcommand{\AnnotationTok}[1]{\textcolor[rgb]{0.56,0.35,0.01}{\textbf{\textit{#1}}}}
\newcommand{\AttributeTok}[1]{\textcolor[rgb]{0.77,0.63,0.00}{#1}}
\newcommand{\BaseNTok}[1]{\textcolor[rgb]{0.00,0.00,0.81}{#1}}
\newcommand{\BuiltInTok}[1]{#1}
\newcommand{\CharTok}[1]{\textcolor[rgb]{0.31,0.60,0.02}{#1}}
\newcommand{\CommentTok}[1]{\textcolor[rgb]{0.56,0.35,0.01}{\textit{#1}}}
\newcommand{\CommentVarTok}[1]{\textcolor[rgb]{0.56,0.35,0.01}{\textbf{\textit{#1}}}}
\newcommand{\ConstantTok}[1]{\textcolor[rgb]{0.00,0.00,0.00}{#1}}
\newcommand{\ControlFlowTok}[1]{\textcolor[rgb]{0.13,0.29,0.53}{\textbf{#1}}}
\newcommand{\DataTypeTok}[1]{\textcolor[rgb]{0.13,0.29,0.53}{#1}}
\newcommand{\DecValTok}[1]{\textcolor[rgb]{0.00,0.00,0.81}{#1}}
\newcommand{\DocumentationTok}[1]{\textcolor[rgb]{0.56,0.35,0.01}{\textbf{\textit{#1}}}}
\newcommand{\ErrorTok}[1]{\textcolor[rgb]{0.64,0.00,0.00}{\textbf{#1}}}
\newcommand{\ExtensionTok}[1]{#1}
\newcommand{\FloatTok}[1]{\textcolor[rgb]{0.00,0.00,0.81}{#1}}
\newcommand{\FunctionTok}[1]{\textcolor[rgb]{0.00,0.00,0.00}{#1}}
\newcommand{\ImportTok}[1]{#1}
\newcommand{\InformationTok}[1]{\textcolor[rgb]{0.56,0.35,0.01}{\textbf{\textit{#1}}}}
\newcommand{\KeywordTok}[1]{\textcolor[rgb]{0.13,0.29,0.53}{\textbf{#1}}}
\newcommand{\NormalTok}[1]{#1}
\newcommand{\OperatorTok}[1]{\textcolor[rgb]{0.81,0.36,0.00}{\textbf{#1}}}
\newcommand{\OtherTok}[1]{\textcolor[rgb]{0.56,0.35,0.01}{#1}}
\newcommand{\PreprocessorTok}[1]{\textcolor[rgb]{0.56,0.35,0.01}{\textit{#1}}}
\newcommand{\RegionMarkerTok}[1]{#1}
\newcommand{\SpecialCharTok}[1]{\textcolor[rgb]{0.00,0.00,0.00}{#1}}
\newcommand{\SpecialStringTok}[1]{\textcolor[rgb]{0.31,0.60,0.02}{#1}}
\newcommand{\StringTok}[1]{\textcolor[rgb]{0.31,0.60,0.02}{#1}}
\newcommand{\VariableTok}[1]{\textcolor[rgb]{0.00,0.00,0.00}{#1}}
\newcommand{\VerbatimStringTok}[1]{\textcolor[rgb]{0.31,0.60,0.02}{#1}}
\newcommand{\WarningTok}[1]{\textcolor[rgb]{0.56,0.35,0.01}{\textbf{\textit{#1}}}}
\usepackage{graphicx,grffile}
\makeatletter
\def\maxwidth{\ifdim\Gin@nat@width>\linewidth\linewidth\else\Gin@nat@width\fi}
\def\maxheight{\ifdim\Gin@nat@height>\textheight\textheight\else\Gin@nat@height\fi}
\makeatother
% Scale images if necessary, so that they will not overflow the page
% margins by default, and it is still possible to overwrite the defaults
% using explicit options in \includegraphics[width, height, ...]{}
\setkeys{Gin}{width=\maxwidth,height=\maxheight,keepaspectratio}
% Set default figure placement to htbp
\makeatletter
\def\fps@figure{htbp}
\makeatother
\setlength{\emergencystretch}{3em} % prevent overfull lines
\providecommand{\tightlist}{%
  \setlength{\itemsep}{0pt}\setlength{\parskip}{0pt}}
\setcounter{secnumdepth}{-\maxdimen} % remove section numbering

\title{Practica\_8.R}
\author{LENGUAJE-R}
\date{2020-12-12}

\begin{document}
\maketitle

\begin{Shaded}
\begin{Highlighting}[]
\CommentTok{###############################################################################}
\CommentTok{# PRACTICA 8 Factores}

\CommentTok{#Cargamos nuestra base de datos}
\NormalTok{bikes <-}\StringTok{ }\KeywordTok{read.csv}\NormalTok{(}\StringTok{"https://raw.githubusercontent.com/AllanZamb/ProcesamientoR/main/BASES/bikes.csv"}\NormalTok{)}

\CommentTok{#Revisar nuestras variables antes de trabajar7}

\CommentTok{#Necesitamos revisar la estructura de nuestras variables}
\KeywordTok{str}\NormalTok{(bikes)}
\end{Highlighting}
\end{Shaded}

\begin{verbatim}
## 'data.frame':    731 obs. of  17 variables:
##  $ X         : int  1 2 3 4 5 6 7 8 9 10 ...
##  $ instant   : int  1 2 3 4 5 6 7 8 9 10 ...
##  $ dteday    : chr  "2011-01-01" "2011-01-02" "2011-01-03" "2011-01-04" ...
##  $ season    : int  1 1 1 1 1 1 1 1 1 1 ...
##  $ yr        : int  0 0 0 0 0 0 0 0 0 0 ...
##  $ mnth      : int  1 1 1 1 1 1 1 1 1 1 ...
##  $ holiday   : int  0 0 0 0 0 0 0 0 0 0 ...
##  $ weekday   : int  6 0 1 2 3 4 5 6 0 1 ...
##  $ workingday: int  0 0 1 1 1 1 1 0 0 1 ...
##  $ weathersit: int  2 2 1 1 1 1 2 2 1 1 ...
##  $ temp      : num  0.344 0.363 0.196 0.2 0.227 ...
##  $ atemp     : num  0.364 0.354 0.189 0.212 0.229 ...
##  $ hum       : num  0.806 0.696 0.437 0.59 0.437 ...
##  $ windspeed : num  0.16 0.249 0.248 0.16 0.187 ...
##  $ casual    : int  331 131 120 108 82 88 148 68 54 41 ...
##  $ registered: int  654 670 1229 1454 1518 1518 1362 891 768 1280 ...
##  $ cnt       : int  985 801 1349 1562 1600 1606 1510 959 822 1321 ...
\end{verbatim}

\begin{Shaded}
\begin{Highlighting}[]
\CommentTok{#Necesitamos revisar el resumen de todas nuestras variables}
\KeywordTok{summary}\NormalTok{(bikes)}
\end{Highlighting}
\end{Shaded}

\begin{verbatim}
##        X            instant         dteday              season     
##  Min.   :  1.0   Min.   :  1.0   Length:731         Min.   :1.000  
##  1st Qu.:183.5   1st Qu.:183.5   Class :character   1st Qu.:2.000  
##  Median :366.0   Median :366.0   Mode  :character   Median :3.000  
##  Mean   :366.0   Mean   :366.0                      Mean   :2.497  
##  3rd Qu.:548.5   3rd Qu.:548.5                      3rd Qu.:3.000  
##  Max.   :731.0   Max.   :731.0                      Max.   :4.000  
##        yr              mnth          holiday           weekday     
##  Min.   :0.0000   Min.   : 1.00   Min.   :0.00000   Min.   :0.000  
##  1st Qu.:0.0000   1st Qu.: 4.00   1st Qu.:0.00000   1st Qu.:1.000  
##  Median :1.0000   Median : 7.00   Median :0.00000   Median :3.000  
##  Mean   :0.5007   Mean   : 6.52   Mean   :0.02873   Mean   :2.997  
##  3rd Qu.:1.0000   3rd Qu.:10.00   3rd Qu.:0.00000   3rd Qu.:5.000  
##  Max.   :1.0000   Max.   :12.00   Max.   :1.00000   Max.   :6.000  
##    workingday      weathersit         temp             atemp        
##  Min.   :0.000   Min.   :1.000   Min.   :0.05913   Min.   :0.07907  
##  1st Qu.:0.000   1st Qu.:1.000   1st Qu.:0.33708   1st Qu.:0.33784  
##  Median :1.000   Median :1.000   Median :0.49833   Median :0.48673  
##  Mean   :0.684   Mean   :1.395   Mean   :0.49538   Mean   :0.47435  
##  3rd Qu.:1.000   3rd Qu.:2.000   3rd Qu.:0.65542   3rd Qu.:0.60860  
##  Max.   :1.000   Max.   :3.000   Max.   :0.86167   Max.   :0.84090  
##       hum           windspeed           casual         registered  
##  Min.   :0.0000   Min.   :0.02239   Min.   :   2.0   Min.   :  20  
##  1st Qu.:0.5200   1st Qu.:0.13495   1st Qu.: 315.5   1st Qu.:2497  
##  Median :0.6267   Median :0.18097   Median : 713.0   Median :3662  
##  Mean   :0.6279   Mean   :0.19049   Mean   : 848.2   Mean   :3656  
##  3rd Qu.:0.7302   3rd Qu.:0.23321   3rd Qu.:1096.0   3rd Qu.:4776  
##  Max.   :0.9725   Max.   :0.50746   Max.   :3410.0   Max.   :6946  
##       cnt      
##  Min.   :  22  
##  1st Qu.:3152  
##  Median :4548  
##  Mean   :4504  
##  3rd Qu.:5956  
##  Max.   :8714
\end{verbatim}

\begin{Shaded}
\begin{Highlighting}[]
\CommentTok{#Trabajamos connversiones de datos de caracteres a factores}

\NormalTok{bikes}\OperatorTok{$}\NormalTok{season <-}\StringTok{ }\KeywordTok{factor}\NormalTok{(bikes}\OperatorTok{$}\NormalTok{season, }
                       \DataTypeTok{levels =} \KeywordTok{c}\NormalTok{(}\DecValTok{1}\OperatorTok{:}\DecValTok{4}\NormalTok{), }
                       \DataTypeTok{labels =} \KeywordTok{c}\NormalTok{(}\StringTok{"Invierno"}\NormalTok{, }\StringTok{"Primavera"}\NormalTok{, }\StringTok{"Verano"}\NormalTok{, }\StringTok{"Otoño"}\NormalTok{) )}


\NormalTok{bikes}\OperatorTok{$}\NormalTok{workingday <-}\StringTok{ }\KeywordTok{factor}\NormalTok{(bikes}\OperatorTok{$}\NormalTok{workingday, }
                           \DataTypeTok{levels =} \KeywordTok{c}\NormalTok{(}\DecValTok{0}\NormalTok{,}\DecValTok{1}\NormalTok{),}
                           \DataTypeTok{labels =} \KeywordTok{c}\NormalTok{(}\StringTok{"No_laboral"}\NormalTok{, }\StringTok{"Laboral"}\NormalTok{))}

\CommentTok{#("Despejado", "Nublado", "LLuvioso")}

\NormalTok{bikes}\OperatorTok{$}\NormalTok{weathersit <-}\StringTok{ }\KeywordTok{factor}\NormalTok{(bikes}\OperatorTok{$}\NormalTok{weathersit, }
                           \DataTypeTok{levels =} \KeywordTok{c}\NormalTok{(}\DecValTok{1}\NormalTok{,}\DecValTok{2}\NormalTok{,}\DecValTok{3}\NormalTok{),}
                           \DataTypeTok{labels =} \KeywordTok{c}\NormalTok{(}\StringTok{"Despejado"}\NormalTok{, }\StringTok{"Nublado"}\NormalTok{, }\StringTok{"LLuvioso"}\NormalTok{))}


\CommentTok{#Vector de colores}
\NormalTok{colores <-}\StringTok{ }\KeywordTok{c}\NormalTok{(}\StringTok{"#B0E9E6"}\NormalTok{, }\StringTok{"#56AD65"}\NormalTok{, }\StringTok{"#C4DC3E"}\NormalTok{, }\StringTok{"#CE8C33"}\NormalTok{)}


\KeywordTok{par}\NormalTok{(}\DataTypeTok{mfrow =} \KeywordTok{c}\NormalTok{(}\DecValTok{1}\NormalTok{,}\DecValTok{1}\NormalTok{))}

\CommentTok{#Categorías }
\KeywordTok{plot}\NormalTok{(bikes}\OperatorTok{$}\NormalTok{workingday, }\DataTypeTok{main =} \StringTok{"Categorías de las dias laborales del año"}\NormalTok{, }
     \DataTypeTok{xlab =} \StringTok{"Temporadas"}\NormalTok{, }
     \DataTypeTok{ylab =} \StringTok{"Frecuencias"}\NormalTok{, }
     \DataTypeTok{col =}\NormalTok{ colores)}
\end{Highlighting}
\end{Shaded}

\includegraphics{Practica_8_files/figure-latex/unnamed-chunk-1-1.pdf}

\begin{Shaded}
\begin{Highlighting}[]
\KeywordTok{plot}\NormalTok{(bikes}\OperatorTok{$}\NormalTok{weathersit, }\DataTypeTok{main =} \StringTok{"Categorías de los climas del año"}\NormalTok{,}
     \DataTypeTok{xlab=} \StringTok{"Climas"}\NormalTok{,}
     \DataTypeTok{ylab =} \StringTok{"Frecuencias"}\NormalTok{, }
     \DataTypeTok{col =}\NormalTok{ colores)}
\end{Highlighting}
\end{Shaded}

\includegraphics{Practica_8_files/figure-latex/unnamed-chunk-1-2.pdf}

\begin{Shaded}
\begin{Highlighting}[]
\KeywordTok{plot}\NormalTok{(bikes}\OperatorTok{$}\NormalTok{season, }\DataTypeTok{main =} \StringTok{"Categorías de las estaciones del año"}\NormalTok{, }
     \DataTypeTok{xlab =} \StringTok{"Temporadas"}\NormalTok{, }
     \DataTypeTok{ylab =} \StringTok{"Frecuencias"}\NormalTok{, }
     \DataTypeTok{col =}\NormalTok{ colores)}
\end{Highlighting}
\end{Shaded}

\includegraphics{Practica_8_files/figure-latex/unnamed-chunk-1-3.pdf}

\begin{Shaded}
\begin{Highlighting}[]
\CommentTok{#Funciones para realizar subconjuntos de datos}

\NormalTok{invierno <-}\StringTok{ }\KeywordTok{subset}\NormalTok{(bikes, season }\OperatorTok{==}\StringTok{ "Invierno"}\NormalTok{)}\OperatorTok{$}\NormalTok{cnt}
\NormalTok{primavera <-}\StringTok{ }\KeywordTok{subset}\NormalTok{(bikes, season }\OperatorTok{==}\StringTok{ "Primavera"}\NormalTok{)}\OperatorTok{$}\NormalTok{cnt}
\NormalTok{verano <-}\StringTok{ }\KeywordTok{subset}\NormalTok{(bikes, season }\OperatorTok{==}\StringTok{ "Verano"}\NormalTok{)}\OperatorTok{$}\NormalTok{cnt}
\NormalTok{otoño <-}\StringTok{ }\KeywordTok{subset}\NormalTok{(bikes, season }\OperatorTok{==}\StringTok{ "Otoño"}\NormalTok{)}\OperatorTok{$}\NormalTok{cnt}

\KeywordTok{par}\NormalTok{(}\DataTypeTok{mfrow =} \KeywordTok{c}\NormalTok{(}\DecValTok{2}\NormalTok{,}\DecValTok{2}\NormalTok{))}

\CommentTok{#Cuantitativas}
\KeywordTok{hist}\NormalTok{(primavera, }\DataTypeTok{main =} \StringTok{"Histograma de renta de bicicletas"}\NormalTok{, }
     \DataTypeTok{xlab =} \StringTok{"Primavera"}\NormalTok{, }
     \DataTypeTok{ylab =} \StringTok{"Frecuencia"}\NormalTok{, }
     \DataTypeTok{col =} \StringTok{"#56AD65"}\NormalTok{, }\DataTypeTok{prob =}\NormalTok{ T )}
\KeywordTok{lines}\NormalTok{(}\KeywordTok{density}\NormalTok{(primavera))}
\KeywordTok{abline}\NormalTok{(}\DataTypeTok{v =} \KeywordTok{mean}\NormalTok{(primavera), }\DataTypeTok{col =} \StringTok{"green"}\NormalTok{)}
\KeywordTok{abline}\NormalTok{(}\DataTypeTok{v =} \KeywordTok{median}\NormalTok{(primavera), }\DataTypeTok{col =} \StringTok{"blue"}\NormalTok{)}



\KeywordTok{hist}\NormalTok{(verano, }\DataTypeTok{main =} \StringTok{"Histograma de renta de bicicletas"}\NormalTok{, }
     \DataTypeTok{xlab =} \StringTok{"Verano"}\NormalTok{, }
     \DataTypeTok{ylab =} \StringTok{"Frecuencia"}\NormalTok{, }
     \DataTypeTok{col =} \StringTok{"#C4DC3E"}\NormalTok{, }\DataTypeTok{prob =}\NormalTok{ T)}
\KeywordTok{lines}\NormalTok{(}\KeywordTok{density}\NormalTok{(verano))}
\KeywordTok{abline}\NormalTok{(}\DataTypeTok{v =} \KeywordTok{mean}\NormalTok{(verano), }\DataTypeTok{col =} \StringTok{"green"}\NormalTok{)}
\KeywordTok{abline}\NormalTok{(}\DataTypeTok{v =} \KeywordTok{median}\NormalTok{(verano), }\DataTypeTok{col =} \StringTok{"blue"}\NormalTok{)}




\KeywordTok{hist}\NormalTok{(otoño, }\DataTypeTok{main =} \StringTok{"Histograma de renta de bicicletas"}\NormalTok{, }
     \DataTypeTok{xlab =} \StringTok{"otoño"}\NormalTok{, }
     \DataTypeTok{ylab =} \StringTok{"Frecuencia"}\NormalTok{,}
     \DataTypeTok{col =} \StringTok{"#CE8C33"}\NormalTok{, }\DataTypeTok{prob =}\NormalTok{ T)}
\KeywordTok{lines}\NormalTok{(}\KeywordTok{density}\NormalTok{(otoño))}
\KeywordTok{abline}\NormalTok{(}\DataTypeTok{v =} \KeywordTok{mean}\NormalTok{(otoño), }\DataTypeTok{col =} \StringTok{"green"}\NormalTok{)}
\KeywordTok{abline}\NormalTok{(}\DataTypeTok{v =} \KeywordTok{median}\NormalTok{(otoño), }\DataTypeTok{col =} \StringTok{"blue"}\NormalTok{)}

\KeywordTok{hist}\NormalTok{(invierno, }\DataTypeTok{main =} \StringTok{"Histograma de renta de bicicletas"}\NormalTok{, }
     \DataTypeTok{xlab =} \StringTok{"Invierno"}\NormalTok{, }
     \DataTypeTok{ylab =} \StringTok{"Frecuencia"}\NormalTok{, }
     \DataTypeTok{col =} \StringTok{"#B0E9E6"}\NormalTok{, }\DataTypeTok{prob =}\NormalTok{ T)}
\KeywordTok{lines}\NormalTok{(}\KeywordTok{density}\NormalTok{(invierno))}
\KeywordTok{abline}\NormalTok{(}\DataTypeTok{v =} \KeywordTok{mean}\NormalTok{(invierno), }\DataTypeTok{col =} \StringTok{"green"}\NormalTok{)}
\KeywordTok{abline}\NormalTok{(}\DataTypeTok{v =} \KeywordTok{median}\NormalTok{(invierno), }\DataTypeTok{col =} \StringTok{"blue"}\NormalTok{)}
\end{Highlighting}
\end{Shaded}

\includegraphics{Practica_8_files/figure-latex/unnamed-chunk-1-4.pdf}

\begin{Shaded}
\begin{Highlighting}[]
\NormalTok{temporada <-}\StringTok{ }\KeywordTok{split}\NormalTok{(bikes, bikes}\OperatorTok{$}\NormalTok{season)}
\end{Highlighting}
\end{Shaded}

\end{document}
